\documentclass [a4paper,12pt]{article} %it used to be 11.5pt
\usepackage{times}
\usepackage{a4,color,palatino,amsmath,amsthm,amssymb,graphicx,setspace}
\usepackage[colon, authoryear]{natbib}
\usepackage{fullpage}
\usepackage[multiple]{footmisc}
\usepackage{tikz}
\usepackage{url}
\usepackage{hyperref}
\usepackage{breakurl}
\usepackage{caption,subcaption}
\usepackage{mhchem}
\usetikzlibrary{patterns}

\pdfinfo{
   /Title  (Is urban densification worth it?)
}
            
\newtheorem{prop}{Proposition}
\newtheorem{ass}{Assumption}
\newtheorem{cor}{Corollary}
\newtheorem{res}{Result}
\newtheorem{rem}{Remark}
\newtheorem{conj}{Conjecture}
\newtheorem{lem}{Lemma}
\DeclareMathOperator*{\argmin}{arg\,min}
\DeclareMathOperator*{\argmax}{arg\,max}

\def\sym#1{\ifmmode^{#1}\else\(^{#1}\)\fi}

\newenvironment{mfignotes}{\begin{footnotesize}\begin{minipage}{\textwidth}\begin{footnotesize}\smallskip\par}
{\vspace{-1em}\end{footnotesize}\end{minipage}\end{footnotesize}}


\linespread{1.5}

\begin{document}


\title{Is urban densification worth it?}
\maketitle

\section*{Introduction} 

As a gross simplification, the debate about the urban form has revolved historically around the centrists, who favour a compact, high-density city, and the decentrists, who argue for spread-out, mostly residential suburbia organised around business cores. The decentrists' position originates form the industrial revolution when theorist argued for a residential model far from the congested and polluted cities, immersed in nature and with abundant land availability \citep{horward1898garden}. The centrists' positions is more recent and sprang from the realisation that cities are expanding at high rate into agricultural and wild land. These different ways of imagining the ideal city, combined with the concern of how to mitigate the effect of human activities on the climate and the environment \citep{COP21}, and a constant increase in the share of population which lives in cities \citep{united2014world}, have fostered spirited debates.

As urban sprawl is mostly considered detrimental to the environment and to the efficient functioning of the city, most administrations have adopted policies to push for densification and brown-field development. The major debate revolves around the optimal urban density in order to guarantee public transports and services and to reduce the consumption of agricultural and wild land. But there are critics who point at the likely increase in traffic congestion and green space reduction, the lack of evidence about sustainability in favour of densification and in general the right of people to choose where and how to live.

My reading of the literature is that, although some serious issues are still to be addressed, there is a strong case for incentivising densification in many parts of the world and in many parts of the cities. Densification is a general principle which has to be tailored to each specific context, but there is no doubt that a certain level of urban density is a necessary condition to provide good and efficient public transports and therefore an advisable goal in order to shift to a more sustainable model. I favour a conservative approach for what concerns the conservation of land and compact cities are less land demanding. Density has also some positive impact on accessibility for people with lower mobility but a negative one on congestion and green space availability. Because the urban form is strictly related to the type of residential settlement and therefore to the type of housing (e.g. flats or single houses), the cultural context should be paramount in designing policies in order to make them acceptable to the public. On the other hand, in my opinion, people tend to stick with what they see and are familiar with, and the duty of public institutions is to foresee problems and point at possible new directions to address them.

In this essay, I first describe urban sprawl and the alternative proposed by the centrists - densification. I then discuss the definition of density that is used both in the literature and in practice. Next, I discuss how urban sprawl and densification impact transports, land conservation and social equality and draw some conclusions. 

\section*{Urban sprawl}

\cite{ewing1997angeles} defines urban sprawl as the ``spread-out, skipped-over development that characterises the non-central city, metropolitan areas and non-metropolitan areas [\ldots]".  It affects most cities in the United States but it is present to different degrees across all continents, including Europe \citep{EuropeanEnvironmentAgency2006} where historically cities developed in a more compact fashion . The spread-out development (both residential and business) is the result of different factors: the congestion of the city centres, the advances in transportation, the rapid growth of private car ownership, the mass production of housing and the availability of capital to buy property and a general preference for the detached house with garden \citep{EuropeanEnvironmentAgency2006, Neuman2005}. 
The problems related to sprawl have long been recognised \citep{RealEstate1974, burchell1998costs, jackson1985crabgrass} and it has been associated with environmental, health, transport and social issues. Sprawling metropolitan development is thought to require more costly infrastructure investments than compact urban forms, to decrease agricultural and wild land availability and biodiversity, to increase air pollution due to higher car dependency and to have less accessibility to services for people with lower mobility.
The extent to which urban sprawl contributes to these issues and whether it contributes at all is still a matter of debate and spread-out cities maintain a wide range of supporters who emphasise the superior quality of life of the suburbia, due to more living space per capita and more access to green space.

\section*{The dense city}

The response to urban sprawl has been almost unanimously a call for a denser city \citep{CommissionEuropeanComm1990, UnitedNations1992, anderson1998uli, national1999our, american1999planning, Eea1999} and densification has become a keyword in various planning guidances across the world \citep{LondonPlan2006, UKGov1994, OECD2012}. 
Densification has been adopted in many theoretical frameworks such as Smart Growth, New Urbanism, Transit Oriented Development (TOD), Compact City, to name a few, as well as a base for many development policies by several city authorities \citep{UKGovPPG3, Rogers2005, council2009vancouver, Dieleman2003}. In the words of \cite{elkin1991reviving} a sustainable city ``must be of a form and scale appropriate to walking, cycling and efficient public transport, and a compactness that encourages social interaction". 

High-density settlements are not new but the concept has gathered new momentum in response to the need for sustainable growth models for cities \citep{dantzig1973compact}. It is argued that a more compact city, with smaller spatial dispersion, is more efficient on many levels and it is also more successful in fuelling economic and social activities \citep{williams1996achieving}. 
In general, the idea of a compact city is not only a physical attribute but it is a broader concept which has stemmed from the work, amongst others, of Jane Jacobs (\citeyear{jacobs1961death}). She argues that diversity is the key to success for the city: within a healthy city, diversity propels networks of social and economic activities which become self-fuelled in a sufficiently complex and interconnected environment. Density is one of the conditions to achieve interconnectivity.


\section*{How density is defined}

At a first glance urban density is a very simple, objective and neutral measure. The analysis of practice and literature, though, reveals a much more variegated landscape. Density is a cross-disciplinary topic and it is used in planning, urban design, architecture, environmental and behavioural studies, transportation, economics, sociology, psychology, anthropology and ecology \citep{churchman1999disentangling}. Its definition varies across disciplines and therefore is difficult to compare across different studies. 
Three concepts are used to describe how density affects urban life: physical density, perceived density and crowding. The latter two are based on the principle that the same environment can be perceived in different ways by different people, in different circumstances. The physical measure of density is defined as ``a point measure defined as the mass of some entity, such as a population of individuals or a collection of buildings described by their size, but normalised by some measure of the area they occupy" \citep{Batty2009}. In order to measure the density of the city, three main indices are used: population density, building density and floor area ratio (or plot ratio). The same area in a city can give different density results depending on whether green areas or roads are included or whether the area is taken as a whole or studied in its parts \citep{campoli2007visualizing}. 

\section*{Transportation and density}

Transport is probably the most important topic for environmental arguments related to urban form and the debate revolves around two main subjects: infrastructure costs and the reduction of car dependency in favour of public transports. 

Infrastructure costs are smaller in a denser city than in a spread-out one due to economies of scale \citep{priest1977large, frank1989costs, burchell1995land}: the length of the network to serve the same amount of people is shorter and fixed costs are amortised over more units. Public transports deliver better performance than private cars in reducing \ce{CO2} emissions \citep{CENTRO2012} and it is generally agreed that a density of about 7500 people/sqKm is needed to support light rail public transport and about 50\% more to support heavy rail \citep{cervero2011urban}. As many cities are pushing for a ``reduction of the external costs of transport by stimulating a modal shift from the private car to public transport" \citep{Nijkamp1996} density seems to be a necessary condition to incentivise this shift but not necessarily sufficient.

On the subject of car dependency, density seems to have some effect in encouraging alternative transportation modes but the contribution is minimal compared to other variables such as fuel prices, household income or the efficiency and capillarity of the public transport network \citep{litman2005land}. Central to this debate is the work of Newman and Kenworthy \citep{newman1992compact, newman1989cities, newman1989gasoline}. They compared ten large US cities and 32 other cities from Europe, Australia and Asia and found that on average, cities with high population density consumed drastically less fuel for traveling. The \cite{ecotec1993reducing} study for the UK Government produced similar results, suggesting that a compact form also reduces the need for traveling. The overall relationship between area per person and gasoline consumption per person is undisputed: if we look at the extremes, it is clear that the level of car usage in the U.S. would be impossible in Hong Kong as well as the level of public transport accessibility of Hong Kong would be economically impossible in most American cities. But the causality is not trivial: it has been suggested that countries which enjoy cheap fuel are more prone to longer and more frequent travels and can therefore afford more dispersed urban forms \citep{gordon1997densities}. Another confounding variable is the possible self-selection: the tendency of people to choose the neighbourhood which facilitates their preferred mode of transport. This was addressed in a study by \cite{naess2009residential}, based on quantitative data and qualitative interviews in Copenhagen, which concludes that a significant relationship between residential location and travel exists beside travel-related preferences. 

Traffic congestion is likely to increase with the increase in urban density. \cite{Melia2011} estimate that although a compact urban form facilitates the shift to more environmentally friendly modes of transport, the effect is less than proportional and therefore an aggravation of traffic congestion is likely to happen.

While density per se does not affect drastically the car dependency or the fuel consumption, it is the sine qua non condition to make public transports economically viable and therefore an important target in achieving sustainability.


\section*{The conservation of land}

The issue of land consumption is simpler in its definition than transport but it is still a matter of debate both in terms of implementation and targets. While the equation more developed land equals less wild or agricultural land is quite straightforward, it is not clear whether the current situation is the very limit or we can still afford urban expansion without compromising the future. Once the limits are set, depending on the local context, the common practice is to limit the development in greenland and create greenbelt areas around cities. This approach has been criticised by \cite{Gordon1997} on the base of the market economy, because restrictions on development force land into lower valued uses. They also argued that land at the moment is not scarce in the U.S. and in the world and that food production has been steadily on the rise in the past decades. This is a valid argument as it is important to make optimal use of the resources but they do not take into account the difficulties in predicting consequences on the long term these changes operate on. I believe a conservative approach should be considered, as modifications to the use of land are very hard to reverse and miscalculations could lead to disasters. 

\section*{Socio-economic issues}

Urban sprawl has been studied from the socio-economic point of view and it is argued that densification has some benefits not only on the transport and land saving side. Densification policies aim at concentrating enough population so that services such as healthcare, cultural activities and social services are economically feasible. Commerce, too, requires a certain concentration of customers and also plays a role in creating a stimulating and diverse environment. 
\cite{popenoe1979urban} identifies sprawl with two types of deprivation costs: deprivation of access and environmental deprivation. The first well documented effect concerns people who have limited mobility (young, elderly and poor) who cannot access community facilities, services and even employment \citep{burchell1993demographic, newman1996reducing}. The second problem is defined by the absence of elements that provide activity and stimulation due to the uniformity of residential, spread-out areas. 
In her analysis of the compact city, \cite{Burton2000} examines the validity of the claim that higher urban density promotes benefits for the low-income groups, in the context of the UK. She finds that for medium-sized English cities, densification has some benefits such as improved public transport, reduced social segregation and better access to facilities at the price of reduced living space and lack of affordable houses. She also finds that a denser city often means reduced access to parks and green areas. Some argue that denser cities have a better access to the country-side but this, in my opinion, is a poor argument as parks and country-side have very different usage. Parks within the city have a more immediate access and need no planning in advance (and no car) to get there. 


\section*{Are the goals achievable or desirable?}

Some critiques to densification focus on whether the goals are achievable or indeed desirable. \cite{breheny1995compact} simulated the total transport energy consumption in Great Britain based on population levels, average mileage per capita and consumption rates per kilometre and then predicted the equivalent energy consumption if no densification had occurred from 1961. He found that this huge effort would save about 2.5\% in weekly national energy consumption, which might be seen as small relatively to the difficulties of implementing the required policies. \cite{Echenique2012}, too, used simulation to test different urban forms on a wide range of performance indicators and found that the transition to ``white-collar" life-style and population growth dominates the impacts on sustainability, rather than urban form. They all suggest that densification to a level that would make an impact would be offset by the gigantic effort required and by the negative consequences due to the low popularity of these policies, the increased congestion levels and the reduction in green spaces.

\cite{Stretton1996} supports the case of the spread-out Australian cities and in response to \cite{newman1992compact} performs an interesting twist of perspective. He states that Australian cities do not perform worse because of higher consumption than European ones; rather they perform better because Australians enjoy four times more green space than Europeans with only 18\% more travel time and 64\% more mileage. He also indirectly raises the question of whether it is right to impose dense urban life on a lot of people who ``voted with their feet" \citep{Neuman2005} to live in suburban areas. 

\cite{Neuman2005} brings forward a very interesting critique to the model of the compact city. Although he does not explicitly disagree with the goals, he sees the compact city model as just a return to outdated models to address new problems. He criticises the idea of the sustainable city in that the form of the city is mostly the outcome of underlying processes, their physical manifestation and states that ``the attempt to attain sustainability via physical means alone is non-sensical". In the words of \cite{batty2013newscience} ``[\ldots] cities must now be looked at as constellations of interactions, communications, relations, flows, and networks, rather than locations [\ldots] location is, in effect, a synthesis of interactions [\ldots]". What he suggests is essentially to focus on the process that cause the city form, rather than the city form which is a consequence of the process. I suppose he is right in a way but we still have to design policies and new bits of cities and it makes sense to try to understand what are the consequences of what we build.


\section*{Conclusion}

As shown above, there is no agreement on the effects of compactness on the sustainability of a city. I suspect that explaining such a complex behaviour as traveling with one variable - density -  which is itself problematic in its definition, is an oversimplification. Even controlling for confounding factors, the question whether densification is a good policy is far too general. The question has to be reframed in the local context, meaning that the reality of what it is achievable, given cultural constraints, has to be taken into account.  Stretton's twist on travel mileage is symptomatic of the fact that not even targets are clear: in his reversed point of view he changes the priorities and defends the current life-style: it is not that Australians consume more, it is just that Australians live better. 

Once that the goals and the boundaries of what is acceptable by the public are determined, I think that policies that incentivise density are advisable because a certain level of urban density is required for efficient public transports, public services and cultural institutions. There is a case for the protection of land, even though the boundaries do not need to be frozen at the present state (again, it is also a matter of location). A compact city does not need to be completely uniform: lower residential areas are also needed to respond to different needs related to different preferences and phases of life of the citizens but these should be developed around transport nodes and be controlled to avoid excess land consumption. Market distortion will arise but will correspond to the value that the public (and therefore politics) attach to the conservation of land. 
In the natural absence of a consensus over life-style, the very important target where everyone should agree is the reduction of emissions and perhaps the UN conference on climate change in Paris is a good step ahead \citep{COP21}. On this ground I believe national and international policies connected to sustainable energy production are required because they have a far bigger impact on reducing emissions rather than urban form.

\clearpage

\bibliographystyle{chicago}
\bibliography{UST_references}

\subsubsection*{Count of words} 
Total words: 2933

\end{document}
 


